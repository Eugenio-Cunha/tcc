%
%Tradução do pacote Algorithm para portugues
%
\renewcommand{\algorithmicrequire}{\textbf{Entrada:}}
\renewcommand{\algorithmicensure}{\textbf{Saída:}}
\renewcommand{\algorithmicend}{\textbf{fim}}
\renewcommand{\algorithmicif}{\textbf{se}}
\renewcommand{\algorithmicthen}{\textbf{então}}
\renewcommand{\algorithmicelse}{\textbf{senão}}
\renewcommand{\algorithmicelsif}{\algorithmicelse \, \algorithmicif}
\renewcommand{\algorithmicendif}{\algorithmicend \, \algorithmicif}
\renewcommand{\algorithmicfor}{\textbf{para}}
\renewcommand{\algorithmicforall}{\textbf{para todo}}
\renewcommand{\algorithmicdo}{\textbf{fazer}}
\renewcommand{\algorithmicendfor}{\algorithmicend \, \algorithmicfor}
\renewcommand{\algorithmicwhile}{\textbf{enquanto}}
\renewcommand{\algorithmicendwhile}{\algorithmicend \, \algorithmicwhile}
\renewcommand{\algorithmicloop}{\textbf{laço}}
\renewcommand{\algorithmicendloop}{\algorithmicend \, \algorithmicloop}
\renewcommand{\algorithmicrepeat}{\textbf{repetir}}
\renewcommand{\algorithmicuntil}{\textbf{até}}
\renewcommand{\algorithmiccomment}[1]{\{#1\}}
\renewcommand{\listalgorithmname}{Lista de Algoritmos}
\floatname{algorithm}{Algoritmo}
%%%%%%%%%%%%%%%%%%%%%%%%%%%%%%%%%%%%%%%%%%%%%%%%%%%%%%%%%%%%%%%%%%%%%%%%%%%%%%%%%%%
\newcommand*\rot{\rotatebox{90}}
\newcommand*\V{\ding{51}}
\newcommand\tab[1][1cm]{\hspace*{#1}}

\makeindex

%%%% O arquivo estiloCAP.tex possui as definições para ciação do estilo de capítulo (fonte de título, barras horizontais, etc.)
% ele não gera texto de saída, é um arquivo de configuração somente
%
% %Estilo de formatação de capítulos

\makeatletter
\newcommand{\thechapterwords}
{ \ifcase \thechapter\or 1\or 2\or 3\or 4\or 5\or6\or 7\or 8\or 9\or 10\or 11\fi}

\def\@makechapterhead#1{%
\vspace*{10\p@}%
{\parindent \z@  \reset@font

\scshape \@chapapp{} \thechapterwords
\quad %
\par\nobreak
\vspace*{10\p@}%
\interlinepenalty\@M
\hrule
\vspace*{10\p@}%
\Huge \bfseries #1\par\nobreak
\par
\vspace*{10\p@}%
\hrule
\vskip 40\p@
}}
\def\@makeschapterhead#1{%
\vspace*{10\p@}%
{\parindent \z@ \centering \reset@font
\par\nobreak
\vspace*{10\p@}%
\interlinepenalty\@M
\hrule
\vspace*{10\p@}%
\Huge \bfseries #1\par\nobreak
\par
\vspace*{10\p@}%
\hrule
\vskip 40\p@
%\vskip 100\p@
}}
%%%%%%%%%%%%%%%%%%%%%%%%%%%%%%%%%%%%%%%%%%%%%%%FIM DO PREAMBULO%%%%%%%%%%%%%%%%%%%%%%%%%%%%%%%%%%%%%%%%%%%%%%%%%%%%%%%%%%%%%%%%%%

