\chapter{Fundamentação Teórica}\label{fund_teo}

\section{Processamento de linguagem natural}
% Como explicado na secção anterior, existem diversas técnicas de processamento de linguagem natural, umas mais específicas e outras mais genéricas. Neste projecto foram implementados dois processos considerados relevantes, tokenization e o POS Tagger (Part of Speech Tagger). O processo de tokenization realiza a divisão do texto em tokens, de modo a facilitar o tratamento posterior de cada token de forma independente. Para a realização desta operação é necessário importar os modelos treinados da ferramenta OpenNLP, que contêm as regras necessárias para a correcta execução dos diversos métodos de processamento.

% Também é determinada a classe gramatical de cada componente das frases analisadas (POS Tagger), sendo necessária a importação do modelo treinado do OpenNLP para uma correta avaliação. Desta forma é possível identificar elementos importantes na frase, como os substantivos e adjectivos, para posteriormente estabelecer a correta polaridade da frase. Segue-se um exemplo5 do resultado deste processo aplicado à uma mensagem específica:

\section{Tokenizer}
\section{Bag of words}
\section{Aprendizado de máquina}
\section{Adaboost}
\section{Naive Bayes}
\section{Testes Scores}
