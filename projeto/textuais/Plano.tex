\chapter{Plano de Trabalho}\label{plano}

Para que seja possível a realização deste estudo, o plano de trabalho consiste em um conjunto de objetivos e processos que visa a conclusão do trabalho de modo adequado. Os pontos relevantes para a realização do trabalho foram ordenados e enumerados nas atividades a serem realizadas:

\section{Plano de atividade}

\begin{enumerate}
\item Revisão Bibliográfica: a primeira atividade enumerada deste estudo, parte da introdução ao problema proposto até a sustentação do estudo por meio de leitura de artigos e trabalhos sobre Aprendizado de Máquina;

\item Coleta de Dados: após a realização da revisão e embasamento dos fundamentos de sustentação do estudo,
a etapa de coleta de dados será executada para formação de um \textit{corpus} de redações;

\item Tratamento dos dados: posteriormente a formação da base de conhecimento, todos os textos de redações serão submetidos individualmente a um processo de normaliazão, este processo visa a remoção de caracteres não alfa numéricos sem alterar o seu valor textual; 

\item Indução do modelo: logo depois de organizar uma base de conhecimento concreta e normalizada, um fluxo de trabalho pode ser projetado e ajustado para induzir um modelo de Aprendizado de Máquina sobre uma porcentagem da base conhecimento que se deseja aprender;

\item Testes e \textit{Score}: imediatamente a indução do modelo, será realizado a etapa de testes de \textit{acurácia} geral, \textit{overfitting} e \textit{noise}, caso algum dos testes falhem ou não atendam aos padrões esperados a etapa anterior será repetida com o objetivo de melhorar os resultados do modelo induzido;

\item Análise dos resultados: após os teste apresentarem um valor aceitável, o restante da base de conhecimento não utilizada na indução do modelo pode ser submetido a predição para comparação gráfica do conhecimento adquirido pelo modelo induzido;

\item Escrita da Monografia: embasado no resultado das etapas anteriores deste trabalho, este processo corrente tem o objetivo de transcrever todos os métodos envolvido para a resolução do problema proposto, considerando as revisões do orientador.

\item Defesa do projeto: por fim, a preparação final para a defesa à banca examinadora.

\end{enumerate}

\section{Cronograma gráfico do Plano de Atividades}

\begin{center}
\rowcolors{1}{white}{gray!25}
\begin{tabular}{c|l|c|c|c|c|c|c|c|c|c|c|}& Atividade & \rot{Fevereiro - 2017} & \rot{Março - 2017} & \rot{Abril - 2017} & \rot{Maio - 2017} & \rot{Junho - 2017} & \rot{Julho - 2017} & \rot{Agosto - 2017} & \rot{Setembro - 2017} & \rot{Outubro - 2017} & \rot{Novembro - 2017} \\
    \hline
    1 & Revisão Bibliográfica   &\V &\V &\V &   &   &   &   &   &   &   \\
    2 & Coleta de Dados         &   &   &   &\V &   &   &   &   &   &   \\
    3 & Tratamento dos dados    &   &   &   &\V &   &   &   &   &   &   \\
    4 & Indução do modelo       &   &   &   &   &\V &\V &\V &\V &   &   \\
    5 & Testes e \textit{Score} &   &   &   &   &\V &\V &\V &\V &   &   \\
    6 & Análise dos resultados  &   &   &   &   &\V &\V &\V &\V &\V &   \\
    7 & Escrita da Monografia   &\V &\V &\V &\V &\V &\V &\V &\V &\V &\V \\
    8 & Entrega da Monografia   &   &   &   &   &   &   &   &   &   &\V \\
    \hline
\end{tabular}
\end{center}

    
