\chapter{Trabalhos Relacionados}\label{trab_rela}


\section{Competências requeridas pela avaliação de redação do enem}

De acordo com ~\cite{silvio_taynan:2017} a prova de Redação do ENEM é avaliada levando em conta uma matriz de referência. Essa matriz, desenvolvida pelo ~\cite{edital_enem:2016}, com a colaboração de especialistas, foi elaborada com o objetivo de operacionalizar o exame. A matriz apresenta cinco competências, para cada competência expressa para redação existem níveis de conhecimento associados de 0 a 5. Isso pode ser visto na tabela abaixo.

\begin{longtable}{|c|l|l|}
    \caption{Matriz de referência elaborada pelo INEP.}
    \endfirsthead
    \multicolumn{3}{l}%
    {Tabela \thetable{} (continuação)}
    \endhead
    \multicolumn{3}{l}%
    {Continua na próxima página}\\
    \endfoot
    \endlastfoot
    \hline
    \multirow{7}{*}{\textbf{I}} & \multicolumn{2}{l|}{\textbf{Demonstrar domínio da norma padrão da língua escrita.}} \\ \cline{2-3} 
     & 0 & \begin{tabular}[c]{@{}l@{}}Demonstra desconhecimento da modalidade escrita formal da \\ língua portuguesa.\end{tabular} \\ \cline{2-3} 
     & 1 & \begin{tabular}[c]{@{}l@{}}Demonstra domínio precário da modalidade escrita formal da \\ língua por-tuguesa, de forma sistemática, com diversificados e \\ frequentes desvios gramaticais, de escolha de registro e de \\ convenções da escrita.\end{tabular} \\ \cline{2-3} 
     & 2 & \begin{tabular}[c]{@{}l@{}}Demonstra domínio insuficiente da modalidade escrita formal \\ da língua portuguesa, com muitos desvios gramaticais, de \\ escolha de registro e de convenções da escrita.\end{tabular} \\ \cline{2-3} 
     & 3 & \begin{tabular}[c]{@{}l@{}}Demonstra domínio mediano da modalidade escrita formal da \\ língua portuguesa e de escolha deregistro, com alguns desvios \\ gramaticais e de convenções da escrita.\end{tabular} \\ \cline{2-3} 
     & 4 & \begin{tabular}[c]{@{}l@{}}Demonstra bom domínio da modalidade escrita formal da língua \\ portuguesa e de escolha de registro,com poucos desvios \\ gramaticais e de convenções da escrita.\end{tabular} \\ \cline{2-3} 
     & 5 & \begin{tabular}[c]{@{}l@{}}Demonstra excelente domínio da modalidade escrita formal da \\ língua portuguesa e de escolha deregistro. Desvios gramaticais \\ ou de convenções da escrita serão aceitos somente como \\ excepcionalidade equando não caracterizem reincidência.\end{tabular} \\ \hline
    \multirow{7}{*}{\textbf{II}} & \multicolumn{2}{l|}{\textbf{\begin{tabular}[c]{@{}l@{}}Compreender a proposta de redação e aplicar conceitos \\ das varias áreas de conhecimento paradesenvolver o tema, \\ dentro dos limites estruturais do texto \\ dissertativo-argumentativo em prosa.\end{tabular}}} \\ \cline{2-3} 
     & 0 & \begin{tabular}[c]{@{}l@{}}``Fuga ao tema/não atendimento à estrutura \\ dissertativo-argumentativa''.\end{tabular} \\ \cline{2-3} 
     & 1 & \begin{tabular}[c]{@{}l@{}}Apresenta o assunto, tangenciando o tema ou demonstra \\ domínio precário do texto dissertativo-argumentativo, com \\ traços constantes de outros tipos textuais\end{tabular} \\ \cline{2-3} 
     & 2 & \begin{tabular}[c]{@{}l@{}}Desenvolve o tema recorrendo à cópia de trechos dos textos \\ motivadores ou apresenta domínioinsuficiente do texto \\ dissertativo-argumentativo, não atendendo à estrutura com \\ proposição, argumentação econclusão.\end{tabular} \\ \cline{2-3} 
     & 3 & \begin{tabular}[c]{@{}l@{}}Desenvolve o tema por meio de argumentação previsível e \\ apresenta domínio mediano do textodissertativo-argumentativo, \\ com proposição, argumentação e conclusão.\end{tabular} \\ \cline{2-3} 
     & 4 & \begin{tabular}[c]{@{}l@{}}Desenvolve o tema por meio de argumentação consistente e \\ apresenta bom domínio do textodissertativo-argumentativo, \\ com proposição, argumentação e conclusão.\end{tabular} \\ \cline{2-3} 
     & 5 & \begin{tabular}[c]{@{}l@{}}Desenvolve o tema por meio de argumentação consistente, a \\ partir de um repertório socioculturalprodutivo e apresenta \\ excelente domínio do texto dissertativo-argumentativo.\end{tabular} \\ \hline
    \multirow{7}{*}{\textbf{III}} & \multicolumn{2}{l|}{\textbf{\begin{tabular}[c]{@{}l@{}}Selecionar, relacionar, organizar e interpretar informações, \\ fatos, opiniões e argumentos em defesa de umponto de vista.\end{tabular}}} \\ \cline{2-3} 
     & 0 & \begin{tabular}[c]{@{}l@{}}Apresenta informações, fatos e opiniões não relacionados \\ ao tema e sem defesa de um ponto devista.\end{tabular} \\ \cline{2-3} 
     & 1 & \begin{tabular}[c]{@{}l@{}}Apresenta informações, fatos e opiniões pouco relacionados \\ ao tema ou incoerentes e sem defesa deum ponto de vista.\end{tabular} \\ \cline{2-3} 
     & 2 & \begin{tabular}[c]{@{}l@{}}Apresenta informações, fatos e opiniões relacionados ao \\ tema, mas desorganizados ou contraditóriose limitados aos \\ argumentos dos textos motivadores, em defesa de um \\ ponto de vista.\end{tabular} \\ \cline{2-3} 
     & 3 & \begin{tabular}[c]{@{}l@{}}Apresenta informações, fatos e opiniões relacionados ao tema, \\ limitados aos argumentos dos textosmotivadores e pouco \\ organizados, em defesa de um ponto de vista.\end{tabular} \\ \cline{2-3} 
     & 4 & \begin{tabular}[c]{@{}l@{}}Apresenta informações, fatos e opiniões relacionados ao tema, \\ de forma organizada, com indícios deautoria, em defesa de \\ um ponto de vista.\end{tabular} \\ \cline{2-3} 
     & 5 & \begin{tabular}[c]{@{}l@{}}Apresenta informações, fatos e opiniões relacionados ao \\ tema proposto, de forma consistente e organizada, configurando \\ autoria, em defesa de um ponto de vista.\end{tabular} \\ \hline
    \multirow{7}{*}{\textbf{IV}} & \multicolumn{2}{l|}{\textbf{\begin{tabular}[c]{@{}l@{}}Demonstrar conhecimento dos mecanismos linguísticos \\ necessários para a construção da argumentação.\end{tabular}}} \\ \cline{2-3} 
     & 0 & Não articula as informações. \\ \cline{2-3} 
     & 1 & Articula as partes do texto de forma precária. \\ \cline{2-3} 
     & 2 & \begin{tabular}[c]{@{}l@{}}Articula as partes do texto, de forma insuficiente, com muitas \\ inadequações e apresenta repertóriolimitado de recursos coesivos.\end{tabular} \\ \cline{2-3} 
     & 3 & \begin{tabular}[c]{@{}l@{}}Articula as partes do texto, de forma mediana, com inadequações, \\ e apresenta repertório poucodiversificado de recursos coesivos.\end{tabular} \\ \cline{2-3} 
     & 4 & \begin{tabular}[c]{@{}l@{}}Articula as partes do texto com poucas inadequações e apresenta \\ repertório diversificado de recursoscoesivos.\end{tabular} \\ \cline{2-3} 
     & 5 & \begin{tabular}[c]{@{}l@{}}Articula bem as partes do texto e apresenta repertório diversificado \\ de recursos coesivos.\end{tabular} \\ \hline
    \multirow{7}{*}{\textbf{V}} & \multicolumn{2}{l|}{\textbf{\begin{tabular}[c]{@{}l@{}}Elaborar proposta de intervenção para o problema abordado, \\ respeitando os direitos humanos.\end{tabular}}} \\ \cline{2-3} 
     & 0 & \begin{tabular}[c]{@{}l@{}}Não apresenta proposta de intervenção ou apresenta proposta não \\ relacionada ao tema ou ao assunto.\end{tabular} \\ \cline{2-3} 
     & 1 & \begin{tabular}[c]{@{}l@{}}Apresenta proposta de intervenção vaga, precária ou relacionada \\ apenas ao assunto.\end{tabular} \\ \cline{2-3} 
     & 2 & \begin{tabular}[c]{@{}l@{}}Elabora, de forma insuficiente, proposta de intervenção relacionada \\ ao tema, ou não articulada com adiscussão desenvolvida no texto.\end{tabular} \\ \cline{2-3} 
     & 3 & \begin{tabular}[c]{@{}l@{}}Elabora, de forma mediana, proposta de intervenção relacionada ao \\ tema e articulada à discussãodesenvolvida no texto.\end{tabular} \\ \cline{2-3} 
     & 4 & \begin{tabular}[c]{@{}l@{}}Elabora bem proposta de intervenção relacionada ao tema e \\ articulada à discussão desenvolvida notexto.\end{tabular} \\ \cline{2-3} 
     & 5 & \begin{tabular}[c]{@{}l@{}}Elabora muito bem proposta de intervenção, detalhada, relacionada \\ ao tema e articulada à discussãodesenvolvida no texto.\end{tabular} \\ \hline
\end{longtable} 

% Este objetivo foi relevante para nosso estudo porque sabemos que as condições de

% 2) classificadores de textos
\section{Modelo de representação de texto mais adequado à classificação}

Segundo a autora ~\cite{alexandra_alves:2010} o BOW (\textit{Bag of Words}) é o modelo mais utilizado em aplicações de classificação de texto. Com baixo custo em termos de processamento este modelo transforma a cadeia de caracteres de um documento num conjunto de palavras, registando além da presença de uma palavra, a sua frequência.

Entretanto propriedades básicas do texto, como a ordem em que as palavras ocorrem e a pontuação, são ignoradas, alem da incapacidade em capturar a semântica do texto, isto é, há palavras com significados distintos que apesar de serem exactamente iguais têm significados diferentes, dependendo do contexto em que são utilizadas.

% falar de stop words
% As stop-words são os termos considerados irrelevantes e somente desempenham um papel
% funcional no texto. São palavras que dependem do idioma e podem também depender do
% tópico de interesse. São normalmente removidas para melhorar os métodos de
% processamento. Em vários casos a remoção das stop-words não traz consequências
% nefastas, palavras como: de, o, a, para, não são úteis dado que têm uma semântica fraca.

Na Tabela 2, \textbf{w}\textsubscript{i} representa uma palavra, \textbf{d}\textsubscript{j} representa um documento e \textbf{p}\textsubscript{ij} o peso atribuido a cada palavra no documento.

\begin{longtable}{|c|c|c|c|c|c|}
    \caption{Modelo \textit{Bag of Words}}
    \endfirsthead
    \multicolumn{6}{l}%
    {Tabela \thetable{} (continuação)}
    \endhead
    \multicolumn{6}{l}%
    {Continua na próxima página}\\
    \endfoot
    \endlastfoot
    \hline  & \textbf{w}\textsubscript{1} & \textbf{w}\textsubscript{2} & \textbf{w}\textsubscript{3} & \textbf{w}\textsubscript{...} & \textbf{w}\textsubscript{n} \\
    \hline \textbf{d}\textsubscript{1} & p\textsubscript{11} & p\textsubscript{12} & p\textsubscript{13} & p\textsubscript{...} & p\textsubscript{1n} \\
    \hline \textbf{d}\textsubscript{2} & p\textsubscript{21} & p\textsubscript{22} & p\textsubscript{23} & p\textsubscript{...} & p\textsubscript{2n} \\
    \hline \textbf{d}\textsubscript{3} & p\textsubscript{31} & p\textsubscript{32} & p\textsubscript{33} & p\textsubscript{...} & p\textsubscript{3n} \\
    \hline \textbf{d}\textsubscript{4} & p\textsubscript{41} & p\textsubscript{42} & p\textsubscript{43} & p\textsubscript{...} & p\textsubscript{4n} \\
    \hline \textbf{d}\textsubscript{n} & p\textsubscript{n1} & p\textsubscript{n2} & p\textsubscript{n3} & p\textsubscript{...} & p\textsubscript{nn} \\
    \hline
\end{longtable}

Ainda segunda a autora existem várias medidas para calcular os valores dos pesos de p\textsubscript{ij}. Essas medidas podem ser classificadas em dois tipos distintos: baseadas em frequências e binárias. Os pesos baseados em frequência visam contabilizar o número de ocorrências de um dado termo num determinado documento, servindo como base para diversas medidas estatísticas e os pesos binários indicam a ocorrência ou não de um dado termo num determinado documento.

\section{Classificadores}

De acordo com ~\cite{porthos_motta:2016} classificadores são utilizados para a predição de classes de objetos e pode ser dita como o processo de generalização dos dados a partir de instâncias diferentes. Existe
uma tendência de se referir a problemas com uma resposta qualitativa (classe) como problemas de classificação e aqueles com uma resposta quantitativas como problemas de
regressão, apesar de nem sempre ser tão simples distinguir isso, pois podemos ter classes
que retornam valores e não dados qualitativos.
% 4) Orange mineração de dados


% A produção escrita treinada em ambiente escolar está concentrada na elaboração de um
% texto conhecido como redação

% A produção textos no ambiente escolar está concentrada na elaboração de redação. Segundo ~\cite{lara:1994}
% foi na segunda metade da década de 70 que se deu o início do processo de redemocratização e a conseqüente crítica da ideologia subjacente à utilização da tecnologia instrucional que restituíram a palavra ao estudante. Nessa mesma época, o decreto de no. 79.298, de 24.02.77 determinou a volta da redação à escola através da``inclusão obrigatória da prova ou questão de redação em língua portuguesa'' nos concursos vestibulares (Art. 1o, alínea d). Assim, para fazer face à nova exigência, a escola passou a enfatizar a produção de textos nas aulas de língua portuguesa, chegando até mesmo a criar disciplinas com o objetivo específico de ensinar a redigir.

% \section{Ferramentas para mineração de dados}

% \section{Modelo emsemble}

% \section{Modelo boost}

% \section{Modelo adaboost}

