\chapter*{Resumo}

\noindent Este trabalho baseia-se na avaliação das competências exigidas em um texto de redação do tipo dissertativo-argumentativo com temas diversificados de ordem social, científica, cultural ou política. Fundamenta-se no estudo das técnicas de Aprendizado de Máquina supervisionado que provê uma gama diversificada de algoritmos poderosos para classificação de textos.

O objetivo deste trabalho é classificar as competências exigidas em um texto de redação do tipo dissertativo-argumentativo a partir do treinamento de um algoritmo de Aprendizado de Máquina, com base em um \textit{corpus} de redações avaliadas.

A compilação de um \textit{corpus} de redações para treinamento e teste de um algoritmo de Aprendizado de Máquina exigiu a prática de extração de informações que compreende técnicas e algoritmos que realiza duas tarefas importantes: a identificação de informações desejadas a partir de documentos estruturados e não-estruturados, e o armazenamento dessas informações em um formato apropriado para uso futuro.

Afim de se avaliar a eficácia dos classificadores, vários experimentos foram executados usando um \textit{corpus} extraído do banco de redações de um serviço que estimula o estudante treinar a produção de textos, em especial do gênero dissertativo-argumentativo. 

O resultado geral de classificação das competências exigidas em um texto de redação obtidas experimentalmente mostraram que o sistema proposto é comparável à avaliação manual de avaliadores capacitados.

\textbf{Palavras-chaves:} Aprendizado de máquina, Banco de Redações, Classificação, Redação