\chapter*{Resumo}

\noindent Este trabalho baseou-se no estudo da avalição de uma redação que é a 
soma de um conjunto de competências exigidas em um texto do tipo 
dissertativo-argumentativo com temas diversificados de ordem social, 
científica, cultural ou política. Fundamentou-se no estudo das técnicas de 
aprendizado de máquina supervisionado que provê uma gama diversificada de 
algoritmos poderosos para classificações de textos.

O objetivo deste trabalho é classificar as competências exigidas em um texto de 
redação a partir do treinamento de um algoritmo de aprendizado de máquina com
base em um \textit{corpus} de redações avaliadas seguindo as competências exigidas em 
uma redação do tipo dissertativa-argumentativa.


% A extração de Informação compreende técnicas e algoritmos que realisam 
% duas tarefas importantes: a identificação de informações desejadas a partir de documentos estruturados e não-estruturados, e o arm
% azenamento dessas informações em um formato apropriado para uso futur
o.
% O \textit{corpus} de redações avaliadas foi minerada de banco de redações manti

Palavras-chaves: Aprendizado de máquina, Redação, Classificação, Orange3
    